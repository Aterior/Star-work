\documentclass[a4paper,11pt,oneside]{scrartcl}

% etwas mehr Platz für Feedback
\usepackage[onehalfspacing]{setspace}


% Standardpakete
\usepackage[utf8]{inputenc}
\usepackage[T1]{fontenc}
\usepackage{microtype}
\usepackage{amsmath}
\usepackage{booktabs}
\usepackage{tikz}
\usepackage{enumitem}
\usepackage{mathpazo}
\usepackage{graphicx}
\usepackage{color}
\usepackage{hyperref}
\usepackage{amssymb}
\usepackage{makecell}

% Sprache wählen
\usepackage[ngerman,shorthands=off]{babel}

% schöner Programmcode
\usepackage{listings}
\lstset{%
  showstringspaces=false,
  mathescape=true,
  inputencoding=utf8,
  numbers=left,
  xleftmargin=\parindent,
  basicstyle=\footnotesize\ttfamily,
  keywordstyle=\bfseries\color{green!40!black},
  commentstyle=\normalfont\itshape\color{black!60},
  identifierstyle=\color{blue},
  stringstyle=\color{orange},
  tabsize=2%
}

\subtitle{ALGO1 · SoSe 2021}

% Ausfüllen:
\title{Lösung zu Aufgabe 8.5}

% Ausfüllen:
\author{%
  Michelangelo Battista <s7657928@stud.uni-frankfurt.de>
  \and Noura Ettalibi <s3638690@stud.uni-frankfurt.de>\and 
Fabian Lebert <s5914579@stud.uni-frankfurt.de>
}
\begin{document}
\maketitle
\section*{Reduktion von 5-Coloring auf 5-careful Coloring}
\fcolorbox{gray!25}{gray!25}{\parbox{\linewidth}{Problemstellung:\\
Zeige, dass das Entscheidungsproblem, ob ein gegebener Graph eine vorsichtige \\ $5$-\textsc{Färbung} besitzt, $\mathcal{NP}-hart$ ist.}}\\\\
$5-Coloring \leq_p 5-careful-Coloring$\\
1) Im Problem \textit{vorsichtige} 5-Färbung, darf es keine Kanten zwischen Knoten geben, deren Färbung sich um weniger als 2 oder mehr als 3 modul 5 unterscheidet.\\
Dementsprechend darf es weder Cliquen der Größe 4 noch der Größe 5 geben.\\
$\Rightarrow$ alle möglichen Cliquen dieser Größe tranformieren, in dem man zwischen Knoten, welche trotz dieser Bedingungen verbunden sind Zwischenknoten einfügt.\\
Für die Knoten gilt in der folgenden Tabelle, die Einträge sind die möglichen Zwischenknoten:\\
\begin{tabular}{c|c|c|c|c|c}
    $\backslash$ & 0 & 1 & 2 & 3 & 4 \\\hline
    0 & X &  \makecell{$3$ \\ $2, 4$} & X & X &  \makecell{$2$ \\ $3, 1$}\\\hline
    1 & \makecell{$3$\\$4, 2$} & X & \makecell{$4$ \\ $3, 0$} & X & X \\\hline
    2 & X & \makecell{$4$ \\ $0, 3$} & X & \makecell{$0$ \\ $4, 1$} & X\\\hline
    3 & X & X & \makecell{$0$ \\ $1, 4$} & X & \makecell{$1$ \\ $0, 2$} \\\hline
    4 & \makecell{$2$ \\ $1, 3$} & X & X & \makecell{$1$ \\ $2, 0$} & X \\
\end{tabular}
\\\\
Dadurch werden alle Kanten, welche der Bedingung nach keiner vorsichtigen Färbung entsprechen,  in Kanten umgewandelt, die nun wieder einer vorsichtigen Färbung entsprechen.\\
Jetzt kann der Graph als Problem im Algorithmus für das Problem "vorsichtige" 5-Färbung eingegeben werden, der Algorithmus, sollte eine vorsichtige 5-Färbung finden,
dadurch dass man nun die hinzugefügten Kanten wieder entfernt, und die alten Kanten wiederherstellt, bekommt man eine 5-Färbung für den gegebenen Graphen.\\
Daraus folgt, dass vorsichtige 5-Färbung mindestens so schwer wie 5-Färbung ist, was bedeutet, dass 5-Färbung NP-schwer ist.
\end{document}